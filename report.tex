%%% Documentation %%%
%
% Pour insérer du code:
% \begin{lstlisting}[title=\textbf{Fichier:} test.c]
% int main(void) {
%		return 0;
% }
% \end{lstlisting}

\documentclass[a4paper]{article}

% Report language (english | french)
\newcommand{\lang}{french}

% Course name
\newcommand{\course}{CSN}

% Assignment
\newcommand{\assignment}{Laboratoire 01}

% Assignment detail
\newcommand{\assignmentDetail}{Affichage linéaire d'une valeur}

% Students
\newcommand{\students}{
	Étudiants: 	& Lucas Elisei \\
        		   			& Arthur Passuello
}

% Professor
\newcommand{\teacher}{
	Professeur: & Etienne Messerli
}

% Assistant
\newcommand{\assistant}{
	Assistant:  & Mike Meury
}

\usepackage{amsmath}
\usepackage[\lang]{babel}
\usepackage{caption}
\usepackage{color}
\usepackage{enumitem}
\usepackage{etoolbox}
\usepackage{fancyhdr}
\usepackage[T1]{fontenc}
\usepackage{geometry}
\usepackage{graphicx}
\usepackage{hyperref}
\usepackage{listings}
\usepackage{tabulary}
\usepackage{tikz}

\geometry{left = 1.0in,right = 1.0in,top = 1.0in,bottom = 1.0in}

% Couleurs pour le code.
\definecolor{pgreen}{rgb}{0.0, 0.5, 0.0}
\definecolor{pred}{rgb}{0.9, 0.0, 0.0}

% Police utilisée pour le code.
\renewcommand{\ttdefault}{pcr}

% Espace avant et après une 'minipage'.
\BeforeBeginEnvironment{minipage}{\vskip 15pt}
\AfterEndEnvironment{minipage}{\vskip 15pt}

% Paramètres du paquet 'listings'.
\lstset{
	backgroundcolor = \color{white},
	basicstyle = \ttfamily,
    	breakatwhitespace = false,
   	breaklines = true,
    	captionpos = t,
    	columns = fixed,
    	commentstyle = \color{pgreen},
    	extendedchars = true,
    	frame = tb,
    	frameround = none,
    	framesep = 0pt,
	framexleftmargin=17pt,
 	framexrightmargin=17pt,
  	framexbottommargin=5pt,
  	framextopmargin=5pt,
    	keywordstyle = \bfseries,
    	numbers = left,
    	numbersep = 10pt,
    	numberstyle = \small\ttfamily,
    	showspaces = false,
   	showstringspaces = false,
   	showtabs = false,
    	stringstyle = \color{pred},
    	tabsize = 2,
    	xleftmargin=19pt,
    	% yes, lstlistings doesn't support UTF-8....
    	literate=
  		{á}{{\'a}}1 {é}{{\'e}}1 {í}{{\'i}}1 {ó}{{\'o}}1 {ú}{{\'u}}1
  		{Á}{{\'A}}1 {É}{{\'E}}1 {Í}{{\'I}}1 {Ó}{{\'O}}1 {Ú}{{\'U}}1
  		{à}{{\`a}}1 {è}{{\`e}}1 {ì}{{\`i}}1 {ò}{{\`o}}1 {ù}{{\`u}}1
	  	{À}{{\`A}}1 {È}{{\'E}}1 {Ì}{{\`I}}1 {Ò}{{\`O}}1 {Ù}{{\`U}}1
  		{ä}{{\"a}}1 {ë}{{\"e}}1 {ï}{{\"i}}1 {ö}{{\"o}}1 {ü}{{\"u}}1
	  	{Ä}{{\"A}}1 {Ë}{{\"E}}1 {Ï}{{\"I}}1 {Ö}{{\"O}}1 {Ü}{{\"U}}1
	  	{â}{{\^a}}1 {ê}{{\^e}}1 {î}{{\^i}}1 {ô}{{\^o}}1 {û}{{\^u}}1
  		{Â}{{\^A}}1 {Ê}{{\^E}}1 {Î}{{\^I}}1 {Ô}{{\^O}}1 {Û}{{\^U}}1
  		{œ}{{\oe}}1 {Œ}{{\OE}}1 {æ}{{\ae}}1 {Æ}{{\AE}}1 {ß}{{\ss}}1
  		{ű}{{\H{u}}}1 {Ű}{{\H{U}}}1 {ő}{{\H{o}}}1 {Ő}{{\H{O}}}1
  		{ç}{{\c c}}1 {Ç}{{\c C}}1 {ø}{{\o}}1 {å}{{\r a}}1 {Å}{{\r A}}1
  		{€}{{\euro}}1 {£}{{\pounds}}1 {«}{{\guillemotleft}}1
  		{»}{{\guillemotright}}1 {ñ}{{\~n}}1 {Ñ}{{\~N}}1 {¿}{{?`}}1
}

% Define code caption style.
\DeclareCaptionFormat{listing}{
	\rule{
		\dimexpr \textwidth+15pt \relax
	}{0.4pt}
	\par \vskip1pt \hspace*{10pt}#1#2#3
}
\captionsetup[lstlisting]{
	format = listing,
	singlelinecheck = false,
	margin=0pt,
	font={sf},
	labelsep=space,
	labelfont=bf
}

% Fait en sorte que le code ne casse pas au milieu d'un saut de page.
\BeforeBeginEnvironment{lstlisting}{\begin{minipage}{\linewidth}}
\AfterEndEnvironment{lstlisting}{\end{minipage}}

\setitemize{itemsep=0.2em}

% Police par défaut.
\renewcommand{\familydefault}{\sfdefault}

\makeatletter
\renewcommand{\@maketitle}{
	\newpage
    	\null
    
    	\begin{tikzpicture}[remember picture,overlay]
    	\node[anchor=north west,inner sep=1cm] at (current page.north west)
    		{\includegraphics[width=0.45\linewidth]{latex/heigvd_logo.png}};
    	\end{tikzpicture}
    
    	\vfill
    
    	\fontfamily{pag}
    
    	\begin{center}
    		\@title
        	\vskip 100pt
        	{\Large \@author}
        	\vskip 60pt
        	{\large \@date}
    	\end{center}
    
    	\vfill
}
\makeatother

\title{
	{\fontsize{40pt}{40pt}\selectfont \course}
    	\vskip 1.5em
    	{\Huge \assignment}
    	\vskip 0.7em
    	{\Large \assignmentDetail}
}

\author{
	\begin{tabular}{ll}
    		\students \\
        	\\
        	\teacher \\
        	\assistant
    \end{tabular}
}

% En-têtes.
\pagestyle{fancy}
\lhead{\course}
\rhead{\assignment}

% Disbable paragraph indentation.
\setlength\parindent{0pt}

% Section numerotation depth.
\setcounter{secnumdepth}{3}

% Table of contents depth.
\setcounter{tocdepth}{3}

% Code language
\lstset{language = vhdl}

%%% Beginning of document.

\begin{document}

\maketitle

\newpage

\tableofcontents

\newpage

\section{Journal de travail}

\subsection*{06.10.2017}

Lors de cette première séance, nous avons tout d'abord analysé le cahier des charges et cherché une décomposition en plusieurs blocs afin de faciliter la compréhension du système.
Nous avons ensuite commencé à coder les différents blocs de notre système.

\subsection*{13.10.2017}

Durant la semaine, nous avons fini de développer le code VHDL. Il est donc temps de tester notre solution. Malheureusement, le bloc \texttt{normal} a posé problème lors de la compilation et le fait que nous ayons utilisé des \lstinline{process} ne facilitait pas le déboguage. Après réflexion, nous avons trouvé une solution combinatoire qui s'est avérée bien plus simple à mettre en place que la prédédente.

Maintenant que le code compile, nous avons pu tester notre solution à l'aide de la simulation ainsi que des \textit{test benchs} et ce fut un plaisir de constater que notre nouvelle approche a porté ses fruits.

\section{Cahier des charges}

\subsection{Mandat}

Le but de ce laboratoire est de réaliser un système permettant l'affichage linéaire d'une valeur.
D'autre part, l'affichage sera piloté afin d'indiquer où se situe cette valeur par rapport à deux bornes \textit{min} et \textit{max}.

Le système comprend 4 modes de fonctionnement:

\begin{center}
\begin{tabular}{|c|c|l|}
	\hline
	\textbf{Commande} & \textbf{Fonction} & \textbf{Description} \\
	\hline
	\texttt{00} & Marche normale & si \textit{valeur} est comprise entre \textit{min} et \textit{max} (bornes incluses): \textbf{TODO} \\
	\hline
	\texttt{01} & Mode linéaire & Affichage linéaire de \textit{valeur}. Les LEDs de 0 à \textit{valeur} sont allumées avec une intensité forte. \\
	\hline
	\texttt{10} & Mode éteint & Toutes les LEDs de l'afficheur sont éteintes. \\
	\hline
	\texttt{11} & Mode allumé & Toutes les LEDs de l'afficheur sont allumées. \\
	\hline
\end{tabular}
\end{center}

\subsection{Spécifications}

Voici l'affichage utilisé pour ce projet (\textit{val} $= 8$, \textit{min} $= 3$, \textit{max} $= 12$):

\begin{minipage}{\textwidth}
	\center
	\includegraphics[width=0.8\textwidth]{figures/specs_ex.png}
	\captionof{figure}{Exemple d'affichage de LEDs}
\end{minipage}

\begin{itemize}
	\item L'intensité forte d'une LED est obtenue par l'enclenchement permanent de celle-ci.
	\item L'intensité faible d'une LED est obtenue par un enclenchement alternatif. La puissance moyenne émise est ainsi plus faible.
	\item Il est garanti que \textit{max} est plus grand que \textit{min}.
\end{itemize}

\subsection{Entrées et sorties du système}

\begin{minipage}{\textwidth}
	\center
	\includegraphics[width=0.8\textwidth]{figures/specs_io.png}
	\captionof{figure}{Schéma bloc du système}
\end{minipage}

\begin{itemize}
	\item \texttt{com\_i } \qquad signal de commande, signal de 2 bits
	\item \texttt{min\_i } \qquad borne inférieure de la plage, signal de 4 bits
	\item \texttt{max\_i } \qquad borne supérieure de la plage, signal de 4 bits
	\item \texttt{osc\_i } \qquad signal d'oscillation pour obtenir une intensité faible, signal de 1 bit
	\item \texttt{val\_i } \qquad valeur d'entrée à visualiser sur l'afficheur linéaire, signal de 4 bits
	\item \texttt{leds\_o} \qquad affichage linéaire composé de 16 LEDs, signal de 16 bits
\end{itemize}

\section{Analyse du système}

\begin{minipage}{\textwidth}
	\center
	\textbf{TODO: Inclure schéma}
	\captionof{figure}{Décomposition en blocs du système}
\end{minipage}

La logique globale du système a été décomposée selon le schéma de la \textbf{Figure 3}. L'analyse de chaque bloc est faite ci-dessous.

\subsection{Partie 1}

\subsubsection{Conception}

\subsubsection{Implémentation}

\subsubsection{Simulation}

\subsubsection{Tests réalisés}

\subsection{Partie 2}

\subsubsection{Conception}

\subsubsection{Implémentation}

\subsubsection{Simulation}

\subsubsection{Tests réalisés}

\section{Conclusion}

\end{document}
